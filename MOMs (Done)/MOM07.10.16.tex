\documentclass{article}
\usepackage[utf8]{inputenc}
\usepackage{graphicx}
\usepackage{multicol}
\usepackage{geometry}
\geometry{textwidth=16cm}
\geometry{textheight=23cm}



\begin{document}

\begin{figure}
\begin{center}
\includegraphics[width=0.13\textwidth]{VAPIQ-PICTURES/Logo2_Tilted.png} %Our logo will be displayed here
\advance\leftskip13.8cm
\end{center}
\end{figure}

\section*{Bachelor Organization Meeting 1}   


\begin{tabular}{ll}                                         \\
\textbf{Date:} 	            & October $7^{th}$ $2016$	    \\
\textbf{Time:}	        	& $09.15-11.15$				    \\
\textbf{Locaton:}       	& Room $2254$, HSN Kongsberg    \\\\
\textbf{Attendees:}         & Stian Fredriksen			    \\
				        	& Tomas Lyngroth			    \\  
				        	& Alekander Holthe 		    	\\
				        	& Vanja Halvorsen		    	\\\\
\textbf{Absent:}		    & None 						    \\\\
\textbf{Minutes taken by:}	& Vanja Halvorsen		        \\\\
\textbf{Topics:}	        & - Final group members         \\	
                            & - Pixhawk                     \\
                            & - Determine fixed meeting day \\
                            & - Determine communication and work tools \\
\end{tabular}                                               \\\\\\\\\\


We have discussed who we would like to join our bachelor team. Based on what we know now about the task given by FFI we have decided that it would be best if one software developer and one electrical engineering student would accompany us. We might have decided on a software developer, but we will discuss this further. Because of circumstances many electrical engineering students are missing one subject and it is worrisome to us to invite them into our team in case they wont be aloud to write the thesis with us. Therefore this will be discussed at a later time.

We want to get our hands on a Pixhawk as soon as possible to explore how it works, and after a conversation with \textbf{Jan Dyre} it was determined that the school will provide us with a \textbf{Pixhawk 2} when it is available to order. Also in this conversation it was mentioned that we will most likely have a meeting with FFI on October $28^{th}$ to discuss the bachelor thesis.

After a discussion and a closer look at our individual schedules it was decided that we will have a fixed meeting day every Friday if nothing else is agreed upon.  

At this meeting we have decided that we want to use \textbf{Jira} as an assignment, issue and project tracking system. We will use \textbf{Drive} to share and store all research and self produced documents and files. From previous conversations we have decided to use \textbf{LaTeX} for typesetting documents. At this meeting we explored how \textbf{ShareLaTeX} works, and this is most likely the platform we will use. 









\section*{Next meeting:}   
\begin{tabular}{ll}                                              
\textbf{Date:} 	            & October $14^{th}$ $2016$	         \\
\textbf{Time:}		        & $11.15-13.15$				         \\
\textbf{Locaton:}	        & Room (XXXX), HSN Kongsberg	     \\\\
\textbf{Topics:}            & - Bachelor name 			         \\
				        	& - Questions for FFI                \\  
				        	& - Cooperation with other group     \\
				        	& - Systems "thinking"	    	     \\
				        	& - Schedule/goals for this semester \\
				        	& - Selecting correct model          \\
				        	

\end{tabular}





\end{document}